\documentclass[11pt]{beamer}
\usetheme{Warsaw}
\usepackage[utf8]{inputenc}
\usepackage[francais]{babel}
\usepackage[T1]{fontenc}
\usepackage{amsmath}
\usepackage{amsfonts}
\usepackage{amssymb}
\usepackage{graphicx}
\author{Clément Tamines, Florent Delgrange}
\title{Statistiques multidimensionnelles}
\subtitle[\ldots]{Arbre de décisions (CART)}

\setbeamercovered{transparent} 
\setbeamertemplate{navigation symbols}{} 
\institute{UMONS\\Faculté des Sciences\\BA 3 Sciences Informatiques}
\date{Juin 2016} 
\subject{Statistiques multidimensionnelles} 
\begin{document}

\begin{frame}
\titlepage
\end{frame}

\begin{frame}
\tableofcontents
\end{frame}

\section{Description de CART}
\subsection{Introduction}
\begin{frame}
\frametitle{Introduction}
Un \textbf{arbre de décision} est un outil de classement de données, les répartissant selon des critères. C'est une méthode \textbf{d'apprentissage}.
\begin{enumerate}
\item Le jeu de données correspond à un vecteur de valeurs.
\item Chaque nœud de l'arbre correspond à un test sur la valeur d'un des attributs.
\item Chaque donnée est répartie selon la valeur booléenne résultante.
\end{enumerate}
Dans ce sens, une branche de l'arbre correspond donc à une décision.
\end{frame}

\subsection{Algorithme}
\begin{frame}
\frametitle{Comment l'arbre est-il construit ?}
Le but est que chaque nœud divise l'ensemble de données en deux sous ensemble les plus homogènes possibles.
Un test est donc associé à un nœud de telle sorte à \textbf{maximiser la réduction d'impureté}.\\
L'impureté d'un échantillon, aussi appelée \textbf{index de Gini}, permet de calculer l'hétérogénéité de celui-ci \textit{(= degré de mélange)}. Pour toute classe $i$, son index Gini est calculé comme suit :
\[Gini(p_i) = p_i (1 - p_i)\]
Plus la probabilité d'une classe $i$ est proche de 0.5, plus l'index est élevé.
\end{frame}

\begin{frame}
\frametitle{Comment l'arbre est-il construit ?}
Le but est que chaque nœud divise l'ensemble de données en deux sous ensemble les plus homogènes possibles.
Un test est donc associé à un nœud de telle sorte à \textbf{maximiser la réduction d'impureté}.\\
\[\Leftrightarrow max \Delta I = p(A)I(A) - p(A_L)I(A_L) - p(A_R)I(A_R)\]
où A est le nœud courant pour un certain test, $A_L$, $A_R$ sont les enfants de gauche et de droite et I(A) est l'impureté du nœud A pour ce test.
\end{frame}

\end{document}