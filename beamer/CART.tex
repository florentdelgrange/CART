\documentclass[11pt]{beamer}
\usetheme{Warsaw}
\usepackage[utf8]{inputenc}
\usepackage[francais]{babel}
\usepackage[T1]{fontenc}
\usepackage{amsmath}
\usepackage{amsfonts}
\usepackage{amssymb}
\usepackage{graphicx}
\author{Clément Tamines, Florent Delgrange}
\title{Statistiques multidimensionnelles}
\subtitle[\ldots]{Arbre de décisions (CART)}

\setbeamercovered{transparent} 
\setbeamertemplate{navigation symbols}{} 
\institute{UMONS\\Faculté des Sciences\\BA 3 Sciences Informatiques}
\date{Juin 2016} 
\subject{Statistiques multidimensionnelles} 
\begin{document}

\begin{frame}
\titlepage
\end{frame}

\begin{frame}
\tableofcontents
\end{frame}

\section{Description de CART}
\subsection{Introduction}
\begin{frame}
\frametitle{Introduction}
Un \textbf{arbre de décision} est un outil de classement de données répartissant celles-ci selon des critères. Chaque nœud de l'arbre correspond à un critère et chaque donnée est répartie selon la valeur booléenne résultante. Dans ce sens, une branche de l'arbre correspond donc à une décision.
\end{frame}

\end{document}